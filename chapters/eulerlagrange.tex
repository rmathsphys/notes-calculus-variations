%!TEX root = ../main.tex
\section{Euler-Lagrange equations}

\subsection{Integrands with only the first derivative}
Consider the functional
\begin{equation}
    S[y] = \int_{x_1}^{x_2} f(x, y(x), y'(x)) dx
    \label{eq:functional-simplest}
\end{equation}
with $f \in C^2$ and $y \in C^2$, and $y(x_1)=y_1, y(x_2)=y_2$. The choice of regularity classes of $f$ and $y$ might appear odd; certainly we only need $f \in C^{0}$ and $y \in C^{1}$ for the equation~\eqref{eq:functional-simplest} to be well-defined. However, later we shall be differentiating $S$, and then the selection of these classes will become clearer.

The function $S$, depends on a sinlge dependent variable $y$. A critical point of $S$ will be a function $y$ that satisfies the boundary conditions, and is sufficietly smooth. The integrand $f$ depends on three variables, and we shall treat them independently from each other.

PROCEDURE.

This is an ordinary differential equation with order (at most) two. This can be seen from expanding the derivative with respect to $x$. PROCEDURE.

To recap this derivative, note that we started by posing the problem and assuming that a critical point $y$ exists. Then, we constructed a suitable class of test functions, $u$, and an auxiliary function $g_u$ that studies the functional around this critical point. By applying results from standard calculus in $\R^n$, and the given boundary conditions, we were able to transform the integrand to a form where the fundamental theorem of calculus of variations can be used. FTCV then allowed us to extract a differential equation from this integrand.

Through this procedure, we have converted the variational problem into an ODE with boundary conditions. The solution of this ODE will be a critical point of the functional. We can determine the nature of each critical point (if any) by substituting it back into the functional. Alternatively, we can perform a \emph{second variation of the functional} and study its behaviour. This route is analogous to the second derivative test from the standard calculus.

EXAMPLE.

EXAMPLE.

\subsection{Elementary analysis of the EL equations}

Briefly introduce first integrals. We will return to these in a later chapter.

\subsection{Higher derivatives}
TBC

\subsection{Multiple dependent variables}
TBC

\subsection{Multiple independent variables}
TBC
